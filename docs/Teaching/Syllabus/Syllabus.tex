\documentclass{article}
\usepackage[utf8]{inputenc}
\usepackage{geometry}
\geometry{a4paper, margin=1in}
\usepackage{enumitem}
\usepackage{hyperref}
\def\author{Maxime Chabriel} \def\instit{ENS de Lyon (CERGIC)} \def\email{maxime.chabriel@proton.me}

\begin{document}
	\par\noindent\rule{\textwidth}{1pt}
	\begin{minipage}[c]{.32\linewidth}
		\vspace{0.5cm}
		\small{ \author \\
			\instit \\
			\href{mailto:\email}{\email} \\
		}
	\end{minipage}
	\begin{minipage}[c]{.33\linewidth}
		\begin{center}
			\large{\textbf{Introduction to Quantitative Methods}} \\
			\normalsize{SCS04144} \\
			\normalsize{Syllabus}
		\end{center}
	\end{minipage}
	\begin{minipage}[c]{.32\linewidth}
		\raggedleft
		\small{ Fall 2024 \\
			Master EEI \\
		}
	\end{minipage}
	\par\noindent\rule{\textwidth}{1pt}
	
	\section*{Course Information}
	
	\begin{itemize}
		\item \textbf{Lecturer:} Maxime Chabriel (Office D4-tbd)
		\item \textbf{Schedule:} Every Thursday (9:00 am-11:00 am), 12 $\times$ 2h
		\item \textbf{Language:} English \footnote{Please feel free to ask if this is an issue for you} 
		\item \textbf{Prerequisites:} An updated version of Excel \footnote{English version. French version or LibreOffice work too but I cannot guaranty full compatibility. In theory, you should be able to download Excel with your ENS Lyon account.} 
		\item \textbf{Evaluation:} Continuous assessment (20\%) + At-home exam (40\%) + Written exam (40\%) 
	\end{itemize}
	
	\section*{Course Description}
	
	This course is designed for M1 students of the Master en études européennes et internationales. Its goal is to provide students with an understanding of the core concepts of quantitative analysis, which they will be able to apply in their future research / working experiences. For the curious or the rigorous mind, some formal mathematical concepts of statistics will be included in the course, but will not be a requirement for final class validation. Each of the concepts learned in class will be put into practice as students will be involved manipulating datasets in Excel (and in R in the final sessions). 
		
	\subsection*{Part 1: Univariate Analysis}
	
	The first part of the course will be devoted to single-variable analysis. We will familiarise ourselves with the concept of distribution and learn how to represent it with statistics and graphical representations. The class will be illustrated with a dataset of variables measuring country levels of development.
	
	\subsection*{Part 2: Multivariate Analysis}
	
	The second part of the course will be devoted to multivariate analysis (comparison tests, simple linear regression, etc.) and their exploitation. There will be a strong focus on the interpretation and caveats of such analysis. Here, the class will be illustrated with a dataset of characteristics of passengers of the Titanic as well as another dataset chosen based on the interests of the students. 
	
	\subsection*{Part 3: Linear Regression and causal analysis}
	
	The third part of the course will be dedicated to linear regression models, causal analysis, and their interpretations. For continuity concerns, we will use the same data as in previous sessions. In the final classes, we will make a transition from Excel to R.
	
	\section*{Evaluation}
	
	The class will be subject to a continuous assessment. Furthermore, one at-home exam will evaluate the students' ability to mobilise a dataset to provide relevant results, analysis and interpretations. A second written exam will evaluate their theoretical knowledge. 
	
\end{document}